\documentclass[a4paper,12pt]{article} % Dokumentklasse und Optionen

% Paket für deutsche Umlaute und Sonderzeichen
\usepackage[utf8]{inputenc}
\usepackage[ngerman]{babel}
\usepackage{amsmath} % Für mathematische Symbole und Umgebungen

% Beginn des Dokuments
\begin{document}

\title{Einführung in LaTeX} % Titel des Dokuments
\author{Max Mustermann}   % Autor
\date{\today}            % Datum

\maketitle % Titel ausgeben

\section{Einleitung}
LaTeX ist ein leistungsstarkes Textsatzsystem, das besonders für wissenschaftliche Arbeiten und technische Dokumente geeignet ist. Mit LaTeX lassen sich \textbf{hochwertige} Dokumente erstellen.

\section{Mathematische Formeln}
LaTeX unterstützt komplexe mathematische Notationen. Zum Beispiel:

\begin{equation}
    E = mc^2
\end{equation}

Man kann auch Summen schreiben:
\[
    \sum_{n=1}^{\infty} \frac{1}{n^2} = \frac{\pi^2}{6}
\]

\section{Aufzählungen}
LaTeX bietet mehrere Möglichkeiten, Aufzählungen zu erstellen:

\begin{itemize}
    \item Punkt 1
    \item Punkt 2\documentclass[a4paper,12pt]{article} % Dokumentklasse und Optionen

    % Paket für deutsche Umlaute und Sonderzeichen
    \usepackage[utf8]{inputenc}
    \usepackage[ngerman]{babel}
    \usepackage{amsmath} % Für mathematische Symbole und Umgebungen
    
    % Beginn des Dokuments
    \begin{document}
    
    \title{Einführung in LaTeX} % Titel des Dokuments
    \author{Max Mustermann}   % Autor
    \date{\today}            % Datum
    
    \maketitle % Titel ausgeben
    
    \section{Einleitung}
    LaTeX ist ein leistungsstarkes Textsatzsystem, das besonders für wissenschaftliche Arbeiten und technische Dokumente geeignet ist. Mit LaTeX lassen sich \textbf{hochwertige} Dokumente erstellen.
    
    \section{Mathematische Formeln}
    LaTeX unterstützt komplexe mathematische Notationen. Zum Beispiel:
    
    \begin{equation}
        E = mc^2
    \end{equation}
    
    Man kann auch Summen schreiben:
    \[
        \sum_{n=1}^{\infty} \frac{1}{n^2} = \frac{\pi^2}{6}
    \]
    
    \section{Aufzählungen}
    LaTeX bietet mehrere Möglichkeiten, Aufzählungen zu erstellen:
    
    \begin{itemize}
        \item Punkt 1
        \item Punkt 2
        \item Punkt 3
    \end{itemize}
    
    Man kann auch nummerierte Listen machen:
    
    \begin{enumerate}
        \item Erster Punkt
        \item Zweiter Punkt
        \item Dritter Punkt
    \end{enumerate}
    
    \section{Abschließende Bemerkungen}
    LaTeX bietet viele weitere Funktionen für Tabellen, Abbildungen, Verweise und mehr. Probieren Sie es aus!
    
    \end{document}
    \documentclass[a4paper,12pt]{article} % Dokumentklasse und Optionen

    % Paket für deutsche Umlaute und Sonderzeichen
    \usepackage[utf8]{inputenc}
    \usepackage[ngerman]{babel}
    \usepackage{amsmath} % Für mathematische Symbole und Umgebungen
    
    % Beginn des Dokuments
    \begin{document}
    
    \title{Einführung in LaTeX} % Titel des Dokuments
    \author{Max Mustermann}   % Autor
    \date{\today}            % Datum
    
    \maketitle % Titel ausgeben
    
    \section{Einleitung}
    LaTeX ist ein leistungsstarkes Textsatzsystem, das besonders für wissenschaftliche Arbeiten und technische Dokumente geeignet ist. Mit LaTeX lassen sich \textbf{hochwertige} Dokumente erstellen.
    
    \section{Mathematische Formeln}
    LaTeX unterstützt komplexe mathematische Notationen. Zum Beispiel:
    
    \begin{equation}
        E = mc^2
    \end{equation}
    
    Man kann auch Summen schreiben:
    \[
        \sum_{n=1}^{\infty} \frac{1}{n^2} = \frac{\pi^2}{6}
    \]
    
    \section{Aufzählungen}
    LaTeX bietet mehrere Möglichkeiten, Aufzählungen zu erstellen:
    
    \begin{itemize}
        \item Punkt 1
        \item Punkt 2
        \item Punkt 3
    \end{itemize}
    
    Man kann auch nummerierte Listen machen:
    
    \begin{enumerate}
        \item Erster Punkt
        \item Zweiter Punkt
        \item Dritter Punkt
    \end{enumerate}
    
    \section{Abschließende Bemerkungen}
    LaTeX bietet viele weitere Funktionen für Tabellen, Abbildungen, Verweise und mehr. Probieren Sie es aus!
    
    \end{document}
    \documentclass[a4paper,12pt]{article} % Dokumentklasse und Optionen

    % Paket für deutsche Umlaute und Sonderzeichen
    \usepackage[utf8]{inputenc}
    \usepackage[ngerman]{babel}
    \usepackage{amsmath} % Für mathematische Symbole und Umgebungen
    
    % Beginn des Dokuments
    \begin{document}
    
    \title{Einführung in LaTeX} % Titel des Dokuments
    \author{Max Mustermann}   % Autor
    \date{\today}            % Datum
    
    \maketitle % Titel ausgeben
    
    \section{Einleitung}
    LaTeX ist ein leistungsstarkes Textsatzsystem, das besonders für wissenschaftliche Arbeiten und technische Dokumente geeignet ist. Mit LaTeX lassen sich \textbf{hochwertige} Dokumente erstellen.
    
    \section{Mathematische Formeln}
    LaTeX unterstützt komplexe mathematische Notationen. Zum Beispiel:
    
    \begin{equation}
        E = mc^2
    \end{equation}
    
    Man kann auch Summen schreiben:
    \[
        \sum_{n=1}^{\infty} \frac{1}{n^2} = \frac{\pi^2}{6}
    \]
    
    \section{Aufzählungen}
    LaTeX bietet mehrere Möglichkeiten, Aufzählungen zu erstellen:
    
    \begin{itemize}
        \item Punkt 1
        \item Punkt 2
        \item Punkt 3
    \end{itemize}
    
    Man kann auch nummerierte Listen machen:
    
    \begin{enumerate}
        \item Erster Punkt
        \item Zweiter Punkt
        \item Dritter Punkt
    \end{enumerate}
    
    \section{Abschließende Bemerkungen}
    LaTeX bietet viele weitere Funktionen für Tabellen, Abbildungen, Verweise und mehr. Probieren Sie es aus!
    
    \end{document}
            
    \item Punkt 3
\end{itemize}

Man kann auch nummerierte Listen machen:

\begin{enumerate}
    \item Erster Punkt
    \item Zweiter Punkt
    \item Dritter Punkt
\end{enumerate}

\section{Abschließende Bemerkungen}
LaTeX bietet viele weitere Funktionen für Tabellen, Abbildungen, Verweise und mehr. Probieren Sie es aus!

\end{document}
