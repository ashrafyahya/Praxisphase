
\documentclass{article}
\usepackage[utf8]{inputenc}
\usepackage{geometry}
\usepackage{hyperref}
\usepackage{graphicx}
\usepackage{longtable}
\geometry{a4paper, margin=1in}

\title{Anweisungen für das Jetson Orin Nano Developer Kit}
\author{}
\date{\today}

\begin{document}

\maketitle

\section{Image auf die microSD-Karte schreiben}

Um Ihr Jetson Orin Nano Developer Kit einzurichten, müssen Sie die microSD-Karte mit dem richtigen Image vorbereiten. Folgen Sie diesen Schritten:

\subsection{1. Das Jetson Orin Nano Developer Kit-Image herunterladen}
\begin{itemize}
    \item Besuchen Sie das \href{https://developer.nvidia.com/embedded/downloads}{Jetson Download Center} und laden Sie das neueste SD-Karten-Image herunter.
    \item Die heruntergeladene Datei liegt normalerweise im \texttt{.zip}-Format vor. Entpacken Sie die Datei, um die \texttt{.img}-Datei zu erhalten.
\end{itemize}

\subsection{2. microSD-Karte in den Host-Computer einlegen}
\begin{itemize}
    \item Verwenden Sie einen microSD-Kartenleser, um die Karte in Ihren Computer einzulegen.
    \item Stellen Sie sicher, dass die microSD-Karte von Ihrem Betriebssystem erkannt wird.
\end{itemize}

\subsection{3. Image auf die microSD-Karte schreiben}
\begin{itemize}
    \item Nutzen Sie ein Tool wie Balena Etcher, das für Windows, Mac und Linux verfügbar ist.
\end{itemize}

\textbf{Schritte mit Balena Etcher:}
\begin{enumerate}
    \item Installieren und öffnen Sie Balena Etcher.
    \item Wählen Sie die entpackte \texttt{.img}-Datei aus.
    \item Wählen Sie die microSD-Karte als Zielgerät aus.
    \item Klicken Sie auf die Schaltfläche \textquotedblleft Flash\textquotedblright, um das Image auf die Karte zu schreiben.
\end{enumerate}

\subsection{4. Image überprüfen}
\begin{itemize}
    \item Balena Etcher überprüft automatisch die geschriebenen Daten.
    \item Wenn die Überprüfung fehlschlägt, wiederholen Sie den Flash-Vorgang oder verwenden Sie eine andere microSD-Karte.
\end{itemize}

\subsection{5. microSD-Karte sicher entfernen}
\begin{itemize}
    \item Werfen Sie die microSD-Karte sicher aus, um Datenbeschädigungen zu vermeiden.
\end{itemize}

\section{Einrichtung und erster Start}

\subsection{1. Verbindungen herstellen}
\begin{itemize}
    \item \textbf{microSD-Karte einlegen:} Setzen Sie die vorbereitete microSD-Karte ein.
    \item \textbf{Peripheriegeräte anschließen:} USB-Tastatur, Maus und Monitor.
    \item \textbf{Netzwerkverbindung herstellen:} Ethernet-Kabel anschließen.
    \item \textbf{Stromversorgung anschließen:} 19V-Netzteil mit dem DC-Eingang verbinden.
\end{itemize}

\subsection{2. Erster Start}
\begin{itemize}
    \item Das Developer Kit startet automatisch. Die grüne LED leuchtet.
    \item Folgen Sie den Anweisungen zur Einrichtung (EULA, Sprache, Zeitzone, WLAN, Benutzerkonto).
\end{itemize}

\subsection{3. Nach dem Login}
\begin{verbatim}
sudo apt update
sudo apt upgrade
\end{verbatim}

\section{Nächste Schritte}
\begin{itemize}
    \item Verwenden Sie den Jetson SDK Manager, um zusätzliche NVIDIA-Softwarepakete zu installieren.
    \item Besuchen Sie die NVIDIA Developer Website für Tutorials und Beispiele.
\end{itemize}

\section{Technische Spezifikationen}

\subsection{Prozessor (CPU)}
\begin{itemize}
    \item \textbf{Architektur:} ARM Cortex-A78AE
    \item \textbf{Kerne:} 6 Kerne, bis zu 1,5 GHz
\end{itemize}

\subsection{Grafikprozessor (GPU)}
\begin{itemize}
    \item \textbf{Architektur:} NVIDIA Ampere, 64 Tensor Cores, 40 TOPS
\end{itemize}

\subsection{Speicher (RAM)}
\begin{itemize}
    \item 4 GB oder 8 GB LPDDR5, bis zu 68 GB/s Bandbreite
\end{itemize}

\subsection{Schnittstellen und I/O}
\begin{itemize}
    \item PCIe Gen 3 x4, USB 3.2, Gigabit-Ethernet
    \item HDMI 2.1, DP 1.2, 2x MIPI CSI-2 Kameraanschlüsse
\end{itemize}

\subsection{Leistung und Energieverbrauch}
\begin{itemize}
    \item Einstellbarer Energieverbrauch: 7 W bis 15 W
\end{itemize}

\subsection{Softwareunterstützung}
\begin{itemize}
    \item Linux for Tegra (L4T), Unterstützung für CUDA 11, C++, Python, TensorFlow, PyTorch
\end{itemize}

\section{Vergleich der Modelle}

\begin{longtable}{|l|c|c|c|}
    \hline
    \textbf{Modell} & \textbf{RAM} & \textbf{Leistung (TOPS)} & \textbf{Energieverbrauch} \\
    \hline
    Jetson Orin Nano 4GB & 4 GB & 20 TOPS & 7–15 W \\
    Jetson Orin Nano 8GB & 8 GB & 40 TOPS & 7–15 W \\
    \hline
\end{longtable}

\end{document}
