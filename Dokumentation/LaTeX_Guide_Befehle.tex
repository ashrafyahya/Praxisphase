
\documentclass{article}
\usepackage[utf8]{inputenc}
\usepackage{geometry}
\usepackage{hyperref}
\usepackage{graphicx}
\usepackage{listings}
\geometry{a4paper, margin=1in}

\title{LaTeX Grundlagen und Wichtige Befehle}
\author{}
\date{\today}

\begin{document}

\maketitle

\section{Grundlagen}
\subsection{Dokumentstruktur}
\begin{verbatim}
\documentclass{article}
\usepackage[utf8]{inputenc}
\usepackage{hyperref}
\begin{document}
% Hier beginnt der Inhalt
\end{document}
\end{verbatim}

\subsection{Zeilenumbruch und Absatz}
\begin{verbatim}
Zeilenumbruch erzwingen: \\

Absatz: Eine Leerzeile im Code einfügen.
\end{verbatim}

\section{Textformatierung}
\subsection{Fett, Kursiv und Unterstreichen}
\begin{verbatim}
\textbf{Fetter Text}
\textit{Kursiver Text}
\underline{Unterstrichener Text}
\end{verbatim}

\subsection{Schriftgröße}
\begin{verbatim}
{\small Kleiner Text} \\
{\Large Großer Text} \\
{\huge Sehr großer Text}
\end{verbatim}

\subsection{Hoch- und Tiefstellen}
\begin{verbatim}
E = mc^2   % Hochstellen mit ^
H_2O       % Tiefstellen mit _
\end{verbatim}

\section{Listen}
\subsection{Nummerierte Liste}
\begin{verbatim}
\begin{enumerate}
    \item Erster Punkt
    \item Zweiter Punkt
\end{enumerate}
\end{verbatim}

\subsection{Aufzählung (Bullet Points)}
\begin{verbatim}
\begin{itemize}
    \item Erster Punkt
    \item Zweiter Punkt
\end{itemize}
\end{verbatim}

\section{Tabellen}
\subsection{Einfache Tabelle}
\begin{verbatim}
\begin{tabular}{|c|c|c|}
\hline
Spalte 1 & Spalte 2 & Spalte 3 \\
\hline
Eintrag 1 & Eintrag 2 & Eintrag 3 \\
\hline
\end{tabular}
\end{verbatim}

\section{Bilder Einfügen}
\begin{verbatim}
\usepackage{graphicx} % im Präambel hinzufügen

\begin{figure}[h]
    \centering
    \includegraphics[width=0.5\textwidth]{bildname.jpg}
    \caption{Beschreibung des Bildes}
    \label{fig:bildlabel}
\end{figure}
\end{verbatim}

\section{Links Einfügen}
\subsection{Externer Link}
\begin{verbatim}
\href{https://www.example.com}{Linktext}
\end{verbatim}

\subsection{Interner Link}
\begin{verbatim}
\label{sec:meinAbschnitt}
\ref{sec:meinAbschnitt}
\end{verbatim}

\section{Fußnoten}
\begin{verbatim}
Hier ist ein Text mit einer Fußnote.\footnote{Das ist eine Fußnote.}
\end{verbatim}

\section{Mathematik}
\subsection{Inline-Mathematik}
\begin{verbatim}
$E = mc^2$
\end{verbatim}

\subsection{Gleichung in eigener Zeile}
\begin{verbatim}
\begin{equation}
    E = mc^2
\end{equation}
\end{verbatim}

\section{Zitate und Literaturverzeichnis}
\begin{verbatim}
\begin{thebibliography}{9}
\bibitem{latex} Lamport, Leslie.
\textit{LaTeX: A Document Preparation System}. Addison-Wesley, 1986.
\end{thebibliography}
\end{verbatim}

\section{Code Einfügen}
\begin{verbatim}
\usepackage{listings}
\begin{lstlisting}
#include <iostream>
int main() {
    std::cout << "Hello, world!";
    return 0;
}
\end{lstlisting}
\end{verbatim}

\newpage
Seitenumbruch erzwingen:

\clearpage
Eine neue Seite beginnen:

% Einrückung 
% Verwenden Sie \hspace für präzise horizontale Abstände.
Verwenden Sie \indent für automatische Einrückungen zu Beginn eines Absatzes.
Verwenden Sie die tabbing-Umgebung für komplexe Textausrichtung mit Tabulatoren.
Verwenden Sie \quad und \qquad für standardisierte horizontale Abstände.

interne Verlinkung
% \usepackage{hyperref}
% \href{relative/pfad/zur/datei.pdf}{Linktext zur Datei}


\end{document}
