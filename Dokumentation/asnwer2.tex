\documentclass[a4paper,12pt]{article}
\usepackage[utf8]{inputenc}
\usepackage[T1]{fontenc}
\usepackage{amsmath}
\usepackage{hyperref}
\usepackage{geometry}
\geometry{a4paper, margin=1in}

\title{Autonomer Betrieb des NVIDIA Jetson Orin Nano Developer Kit mit Batterie oder Powerbank}
\author{}
\date{}

\begin{document}

\maketitle

\section*{1. Strombedarf des Jetson Orin Nano Developer Kit}

Das Jetson Orin Nano Developer Kit 8GB benötigt \textbf{19V} Eingangsspannung und hat eine typische Leistungsaufnahme von:

\begin{table}[h!]
    \centering
    \begin{tabular}{|l|c|}
        \hline
        \textbf{Betriebsmodus} & \textbf{Leistungsaufnahme} \\
        \hline
        Leerlauf & 5--7 W \\
        Moderate Last & 7--10 W \\
        Maximale Leistung & 15--20 W \\
        \hline
    \end{tabular}
    \caption{Typische Leistungsaufnahme des Jetson Orin Nano.}
\end{table}

Für dauerhafte Objekterkennung ist ein Verbrauch von etwa \textbf{15 W bis maximal 20 W} realistisch, abhängig von der GPU- und CPU-Auslastung.

\subsection*{Stromanschluss}
\begin{itemize}
    \item \textbf{Eingangsspannung:} 19V DC.
    \item \textbf{Anschlussart:} Barrel-Jack (5,5 mm außen, 2,1 mm innen).
\end{itemize}

\section*{2. Energiequelle: Batterie oder Powerbank}

\subsection*{a. Batterie (empfohlen für längere Laufzeiten)}

Eine Batterie mit \textbf{19V Ausgang} ist ideal für das Jetson Orin Nano. Falls die Batterie eine andere Spannung liefert (z. B. 12V), wird ein \textbf{DC-DC Spannungswandler} benötigt.

\subsubsection*{1. Direkte 19V-Batterien}
\begin{itemize}
    \item \textbf{Typ:} Lithium-Ionen- oder Lithium-Polymer-Batterien.
    \item \textbf{Kapazität:} Mindestens 20Ah für längere Laufzeiten.
\end{itemize}

\paragraph{Beispielrechnung:}
\begin{itemize}
    \item \textbf{Batterie:} 19V 10Ah $\approx$ 190 Wh.
    \item \textbf{Verbrauch:} 15 W.
\end{itemize}
\[
    \text{Betriebsdauer} = \frac{\text{Kapazität in Wh}}{\text{Leistungsaufnahme}} = \frac{190}{15} \approx 12,6 \text{ Stunden.}
\]

\subsubsection*{2. Batterien mit Spannungswandler}
Falls die Batterie nicht direkt 19V liefert:
\begin{itemize}
    \item \textbf{Typische Batterie:} 12V oder 24V (z. B. LiFePO4 oder Blei-Säure).
    \item \textbf{Spannungswandler:} Ein \textbf{Step-Up (12V $\to$ 19V)} oder \textbf{Step-Down (24V $\to$ 19V)} DC-DC-Wandler.
\end{itemize}

\paragraph{Empfehlung:}
\begin{itemize}
    \item \textbf{Spannungswandler:} DROK DC-DC-Wandler (12V $\to$ 19V, 5A) oder ähnliche.
    \item \textbf{Batterie:} 12V 20Ah (Li-Ion) oder größere Kapazitäten.
\end{itemize}

\subsection*{b. Powerbank mit 19V Unterstützung}

Eine Powerbank ist einfacher zu handhaben, erfordert jedoch, dass sie einen \textbf{19V-Ausgang} bietet. Standard-USB-PD (Power Delivery) reicht \textbf{nicht aus}, da USB-PD normalerweise maximal 15V oder 20V liefert.

\paragraph{Anforderungen:}
\begin{itemize}
    \item \textbf{19V-Ausgang:} Über Barrel-Jack oder Adapterkabel.
    \item \textbf{Kapazität:} Mindestens 20.000 mAh bei 19V ($\approx$ 76 Wh).
\end{itemize}

\paragraph{Beispielrechnung:}
\begin{itemize}
    \item \textbf{Powerbank:} 76 Wh.
    \item \textbf{Verbrauch:} 15 W.
\end{itemize}
\[
    \text{Betriebsdauer} = \frac{76}{15} \approx 5 \text{ Stunden.}
\]

\paragraph{Empfehlungen:}
\begin{itemize}
    \item \textbf{Omnicharge Omni 20+:} Bietet 19V DC-Ausgang und USB-C PD (45W).
    \item \textbf{MaxOak K2:} Große Kapazität (50.000 mAh) und 19V DC-Ausgang.
\end{itemize}

\section*{3. Optimierung des Energieverbrauchs}

\subsection*{a. Energiemodi des Jetson Orin Nano}

\begin{itemize}
    \item \textbf{Maximale Leistung (Standard):}
    \begin{verbatim}
    sudo nvpmodel -m 0
    \end{verbatim}
    \item \textbf{Energiesparmodus:}
    \begin{verbatim}
    sudo nvpmodel -m 1
    \end{verbatim}
\end{itemize}

\subsection*{b. Nutzung von \texttt{tegrastats}}
Überwache den aktuellen Energieverbrauch mit:
\begin{verbatim}
tegrastats
\end{verbatim}

\subsection*{c. Peripheriegeräte entfernen}
Schalte nicht benötigte USB-Geräte, Monitore oder Kameras ab, um den Energieverbrauch zu senken.

\section*{4. Fazit}

\begin{itemize}
    \item \textbf{Typischer Verbrauch bei Objekterkennung:} 15--20 W.
    \item \textbf{Powerbank-Lösung:} Praktisch für kürzere Laufzeiten, z. B. 4--6 Stunden.
    \item \textbf{Batterie-Lösung:} Ideal für lange Laufzeiten, z. B. bis zu 12--16 Stunden.
    \item \textbf{19V direkt erforderlich:} Entweder über passende Powerbank oder eine Batterie mit DC-DC-Wandler.
\end{itemize}

Falls du weitere Details benötigst, z. B. zur Einrichtung von Spannungswandlern oder konkreten Modellen, lass es mich wissen!

\end{document}
