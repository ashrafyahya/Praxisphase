\documentclass[12pt]{article}
\usepackage{amsmath}

\title{Autonome Stromversorgung des Jetson Orin Nano für Objekterkennung}
\author{}
\date{}

\begin{document}

\maketitle

\section{Einleitung}
Um den NVIDIA Jetson Orin Nano autonom zu betreiben, ist eine geeignete Stromquelle erforderlich, die sowohl die Anforderungen des Jetson-Boards als auch den Energieverbrauch während der dauerhaften Objekterkennung deckt. In dieser Arbeit werden die Energieanforderungen des Jetson Orin Nano untersucht, mögliche Batterien und Powerbanks zur Stromversorgung vorgestellt und berechnet, wie lange der Jetson Orin Nano bei kontinuierlicher Objekterkennung betrieben werden kann.

\section{Strombedarf des Jetson Orin Nano}
Der Strombedarf des Jetson Orin Nano hängt von der Auslastung des Systems ab. Im Allgemeinen wird er in folgenden Bereichen liegen:

\begin{itemize}
    \item \textbf{Leerlauf}: 5-7 W
    \item \textbf{Moderate Last}: 7-10 W
    \item \textbf{Maximale Leistung}: 15 W (mit Peripheriegeräten bis zu 20 W)
\end{itemize}

Für Anwendungen wie die Objekterkennung (z.B. YOLO, SSD) wird der Verbrauch typischerweise im Bereich von 10 bis 15 W liegen.

\subsection{Stromversorgung}
Der Jetson Orin Nano benötigt eine stabile Eingangsspannung von \textbf{5V} (über USB-C) oder \textbf{12V-19V} (über Barrel-Jack). Der typische Strombedarf beträgt \textbf{3A bis 4A} bei einer Spannung von 5V.

\section{Energiequelle: Batterie oder Powerbank}

\subsection{Powerbank}
Eine Powerbank ist eine einfache Lösung, um den Jetson Orin Nano autonom zu betreiben. Die Powerbank sollte folgende Eigenschaften aufweisen:
\begin{itemize}
    \item \textbf{Unterstützung von USB-PD (Power Delivery)}: Sie sollte mindestens 15W bis 20W bei 5V oder 9V liefern können.
    \item \textbf{Kapazität}: Um den Jetson für längere Zeit autonom zu betreiben, sollte die Powerbank eine hohe Kapazität besitzen.
\end{itemize}

\subsubsection{Beispielrechnung für Powerbank}
Wenn die Powerbank eine Kapazität von 20.000 mAh (3.7V) ≈ 74 Wh hat und der Jetson Orin Nano 15 W verbraucht, lässt sich die Betriebsdauer wie folgt berechnen:
\[
\text{Betriebsdauer} = \frac{\text{Kapazität in Wh}}{\text{Leistungsaufnahme}} = \frac{74}{15} \approx 4,9 \, \text{Stunden}
\]
Empfohlene Modelle: Powerbanks mit USB-C PD und mindestens 20.000 mAh.

\subsection{Batterie}
Eine Li-Ion oder LiFePO4-Batterie bietet eine weitere Möglichkeit, den Jetson Orin Nano autonom zu betreiben. Typische Parameter einer solchen Batterie:
\begin{itemize}
    \item \textbf{Spannung}: 12V bis 19V.
    \item \textbf{Kapazität}: z.B. 12V 20Ah ≈ 240 Wh.
    \item \textbf{Vorteil}: Lange Laufzeit und Wiederaufladbarkeit.
\end{itemize}

\subsubsection{Beispielrechnung für Batterie}
Wenn die Batterie eine Kapazität von 240 Wh hat und der Jetson Orin Nano 15 W verbraucht, lässt sich die Betriebsdauer wie folgt berechnen:
\[
\text{Betriebsdauer} = \frac{240}{15} \approx 16 \, \text{Stunden}
\]

\section{Optimierung der Energieeffizienz}

\subsection{Energiemodi des Jetson Orin Nano}
Der Jetson Orin Nano bietet verschiedene Energiemodi, um den Energieverbrauch zu steuern:
\begin{itemize}
    \item \textbf{Standardleistung}: 
    \begin{verbatim}
    sudo nvpmodel -m 0
    \end{verbatim}
    Maximale Leistung (ca. 15 W).
    
    \item \textbf{Energiesparmodus}: 
    \begin{verbatim}
    sudo nvpmodel -m 1
    \end{verbatim}
    Reduziert den Verbrauch auf etwa 7-10 W.
    
    \item \textbf{Taktrate anpassen}: 
    \begin{verbatim}
    sudo jetson_clocks --show
    sudo jetson_clocks --store
    \end{verbatim}
    Optimiert die Taktraten für den Energieverbrauch.
\end{itemize}

\subsection{Peripheriegeräte}
Reduziere den Energieverbrauch, indem du nicht benötigte Peripheriegeräte (z.B. Monitor, Tastatur, USB-Geräte) abschaltest.

\section{Empfohlene Komponenten für den Betrieb}

\subsection{Powerbank}
Folgende Powerbanks sind geeignet:
\begin{itemize}
    \item \textbf{Anker PowerCore+ 26800 PD} (45W max, USB-C): Etwa 6-8 Stunden Laufzeit.
    \item \textbf{Zendure SuperTank Pro} (100W max, USB-C): Bis zu 10 Stunden Laufzeit.
\end{itemize}

\subsection{Batterie mit Spannungswandler}
Für längeren Betrieb:
\begin{itemize}
    \item \textbf{Batterie}: Li-Ion-Batterie, z.B. 12V 20Ah oder 19V 10Ah.
    \item \textbf{Spannungswandler}: DC-DC-Wandler (z.B. DROK Step-Down Converter).
\end{itemize}

\section{Messung des Energieverbrauchs}
Um den tatsächlichen Energieverbrauch des Jetson Orin Nano zu messen, kann man folgende Tools verwenden:
\begin{itemize}
    \item \textbf{Software-Tools auf dem Jetson}:
    \begin{verbatim}
    tegrastats
    \end{verbatim}
    Dieses Tool zeigt die CPU-, GPU- und RAM-Auslastung sowie den Energieverbrauch in Echtzeit.

    \item \textbf{Externe Messgeräte}: 
    Verwende ein USB-Leistungsmessgerät oder einen DC-Wattmeter, um den Energieverbrauch präzise zu messen.
\end{itemize}

\section{Fazit}
Für kurze Tests reicht eine Powerbank mit mindestens 20.000 mAh und USB-C PD aus. Für längeren autonomen Betrieb empfiehlt sich eine Batterie mit einer Kapazität von 12V 20Ah oder 19V 10Ah. Der typische Verbrauch des Jetson Orin Nano bei Objekterkennung liegt bei 10-15 W, was mit einer Powerbank von 20.000 mAh etwa 4-6 Stunden Laufzeit ergibt. Eine 12V 20Ah Batterie könnte bis zu 16 Stunden Betrieb ermöglichen.

\end{document}
