
\documentclass{article}
\usepackage[utf8]{inputenc}
\usepackage{geometry}
\usepackage{hyperref}
\geometry{a4paper, margin=1in}

\title{Einrichtungserrors}
\author{}
\date{\today}

\begin{document}

\maketitle

\section{Genutzte Hardware für Image-Flashen}

Folgende Software wurde auf zwei Laptops ausgeführt. Der eine ist ein Hochschul-Laptop, den ich für das C++-Tutorium nutze. Der andere ist mein persönlicher Laptop.

\subsection{Hochschul-Laptop}
\begin{itemize}
    \item \textbf{Betriebssystem}: Windows 10
    \item \textbf{Speicher}: 237GB
    \item \textbf{RAM}: 8GB
    \item \textbf{Weiteres}: Core i3, 2 Kerne, 4 logische Prozessoren
\end{itemize}

\subsection{Persönlicher Laptop}
\begin{itemize}
    \item \textbf{Betriebssystem}: Windows 10
    \item \textbf{Speicher}: 222GB SSD + 1TB DDR
    \item \textbf{RAM}: 16GB
    \item \textbf{Weiteres}: Core i5, 4 Kerne, 8 logische Prozessoren
\end{itemize}

\newpage
\section{Errors-Liste und mögliche Lösungen und Empfehlungen}
\begin{itemize}
    \item \textbf{SD Card Formatter}: \\Funktioniert nur auf dem persönlichen PC. Auf dem Hochschul-Laptop wird die SD-Karte nicht erkannt.
    \item \textbf{Etcher}: \\Hängt auf dem Hochschul-Laptop. Auf meinem persönlichen PC führt es beim Flashen zum Blue Screen.
    \item \textbf{Rufus}: \\Funktioniert meist, aber manchmal gibt es Blue Screens beim Formatieren der SD-Karte.
    \item \textbf{Win32 Disk Manager}: \\Erkennt auf meinem PC die SD-Karte nicht. Auf dem Hochschul-Laptop wird die Karte erkannt, aber der Flash-Prozess startet nicht.
    \item \textbf{Flash-Prozess}: \\Funktioniert bei Rufus meist, einmal auch bei Etcher. Das Image wird jedoch nicht korrekt geflasht, und das System bootet nicht korrekt.
    \item \textbf{System bootet nicht}: \\Problem tritt auf, da das System JetPack 6.1 nicht unterstützt.
    \item \textbf{SDK Manager}: \\Empfehlung, auf Linux zu arbeiten. Installationsanleitung: VirtualBox, Ubuntu, 35GB Mindestspeicher.
    \item \textbf{Speicherprobleme in VirtualBox}: \\Speicher außerhalb der virtuellen Maschine vergrößern.
    \item \textbf{Partitionierung}: \\SDK Manager nutzt möglicherweise den falschn falschen Datenträger. Verwenden Sie GParted zur Größenanpassung.
    \item \textbf{USB-Verbindung}: \\USB in VirtualBox unter \texttt{Settings} hinzufügen. Achten Sie darauf, dass der Name der NVIDIA-Hardware angezeigt wird.
    \item \textbf{Recovery Mode}: \\Verbinden Sie die PINS \texttt{Ground} und \texttt{REC} mit Female-to-Female-Jumper-Kabeln.
    \item \textbf{Gerät wird nicht erkannt}: \\Stromversorgung überprüfen und Gerät im Recovery Mode starten.
    \item \textbf{Virtuelle Maschine friert ein}: \\Speicher erhöhen und virtuelle Maschine neu installieren.
    \item \textbf{Schwarzer Bildschirm}: \\Verwenden Sie den Ubuntu-Befehl \texttt{dd}, um das Image auf die SD-Karte zu flashen.
\end{itemize}

\end{document}
